\documentclass[letterpaper]{article}
\usepackage{url,multirow}
\usepackage{hyperref}
\usepackage{subscript}
\usepackage{cite}
\usepackage{graphicx}
\usepackage{colortbl, xcolor}
\usepackage{todonotes}
\usepackage{caption}
\usepackage{subcaption}
\usepackage{graphicx}
\usepackage{booktabs}
\usepackage{comment}
\usepackage{flushend}
\usepackage{xspace}
\usepackage{booktabs}
\usepackage{mdwlist}
\usepackage{array}
\usepackage[justification=centering]{caption}

\newcolumntype{N}{>{\centering\arraybackslash}m{.7in}}
\newcolumntype{G}{>{\centering\arraybackslash}m{2in}}

\newcommand*{\boldsymbol}[1]{#1}
\newcommand\tab[1][1cm]{\hspace*{#1}}

\begin{document}

\title{title}
\author{authors}
\maketitle

\begin{abstract}
abstract
\end{abstract}

\section{Introduction}
\section{Problem statement}
\section{Our scheme}
\subsection{Architecture/pipeline}
[Matt]
\subsection{Detectors}
\subsubsection{X-Mailer header}
\tab The X-Mailer header denotes the type of software or mail client that a sender used to send the message. The value of the X-Mailer header often depends on the device used to send the message (and not just the domain part of the email address). For example, an email address with a domain part of	``gmail.com'' might have as its corresponding X-Mailer value ``iPhone Mail (12B466)''. Since the software and device used is a fundamental component of a sender's profile, it has the potential to act as sort of identity marker of a sender. As such, the detector constructed for this header relies on the consistency of the field throughout emails sent by the same sender.\\
\tab The detector's sender profile keeps track of a mapping from senders (email addresses extracted from the From header) to a list of parsed X-Mailer values. However, since email clients update versions frequently, the detector attempts to take out any version numbers. For example, the parsed value of the X-Mailer value ``iPhone Mail (12B466)'' would be ``iPhone Mail''. In addition, if no such X-Mailer header exists in the email, the sender profile will instead append the python value ``None'' to the list of X-Mailer values. With this sender profile, the detector then is able to classify emails by the following process:

\begin{enumerate}
\item Extract the email address from the From header and the parsed value from the X-Mailer value (which is ``None'' if no such X-Mailer header exists in the email's headers).
\item Look up in the sender profile the corresponding sender's list of parsed X-Mailer values.
\item If the parsed X-Mailer value of the email is in the corresponding sender's list of parsed X-Mailer values (ignoring case in comparison), the email is classified as safe.
\item Otherwise, it is classified as phishing.
\end{enumerate}

\tab In the email dataset used in the Evaluation section, it was found that 97.75\% of senders remained consistent with their X-Mailer headers over time (i.e. the list tracked in the sender profile for a given sender had only one unique value of an X-Mailer in it).  Through this classification method and consistency, the detector exploits any lack of consistency of the X-Mailer header for a given sender.  However, the effects of the detector are limiting due to the sparseness of X-Mailer headers in emails (roughly 83.56\% of emails in the dataset used in the Evaluation). Since only 16.43\% of emails have X-Mailer headers, the detector will falsely classify many emails as safe when indeed they are phishing. For this reason, the detection rate (as well as the false alarm rate) is significantly low. Thus, the detector is mostly effective when it is classifying emails from senders that send emails that generally contain X-Mailer values, where the percentage of senders that consistently send the same X-Mailer is 97.05\%.





\subsubsection{Order of headers}
\subsubsection{Message-ID header}
\subsubsection{Date header}
\subsubsection{Timezone}
\subsubsection{Content-Type header}
\subsubsection{Content-Transfer-Encoding header}
\section{Evaluation}
\subsection{Personal emails}
[Michael]

\begin{table}[]
\centering
\begin{tabular}{N|N|N|N|N|N|N|}
\cline{2-7}
     & \multicolumn{2}{c|}{\textbf{1:2 Weights}} & \multicolumn{2}{c|}{\textbf{1:10 Weights}} & \multicolumn{2}{c|}{\textbf{1:100 Weights}} \\  \hline
\multicolumn{1}{|c|}{\textbf{Features}} & \textbf{False Alarm (\%)} & \textbf{Detection (\%)} & \textbf{False Alarm (\%)} & \textbf{Detection (\%)} & \textbf{False Alarm (\%)} & \textbf{Detection (\%)}  \\ \hline
\multicolumn{1}{|c|}{X-Mailer} & 0.xxx & 0.xxx & 0.xxx & 0.xxx & 0.xxx & 0.xxx  \\ \hline
\multicolumn{1}{|c|}{Received} & 0.xxx & 0.xxx  & 0.xxx & 0.xxx & 0.xxx & 0.xxx  \\ \hline
\multicolumn{1}{|c|}{Order of Headers} & 0.xxx & 0.xxx  & 0.xxx & 0.xxx & 0.xxx & 0.xxx  \\ \hline
\multicolumn{1}{|c|}{Content Type} & 0.xxx & 0.xxx  & 0.xxx & 0.xxx & 0.xxx & 0.xxx  \\ \hline
\multicolumn{1}{|c|}{Date Format} & 0.xxx & 0.xxx  & 0.xxx & 0.xxx & 0.xxx & 0.xxx  \\ \hline
\multicolumn{1}{|c|}{Message ID Domain} & 0.xxx & 0.xxx  & 0.xxx & 0.xxx & 0.xxx & 0.xxx  \\ \hline
\multicolumn{1}{|c|}{Date Timezone} & 0.xxx & 0.xxx  & 0.xxx & 0.xxx & 0.xxx & 0.xxx  \\ \hline
\end{tabular}
\caption{False Alarm and Detection Results from Varying Weights and Features}
\end{table}


\begin{table}[]
\centering
\begin{tabular}{N|N|N|N|N|N|N|}
\cline{2-7}
     & \multicolumn{2}{c|}{\textbf{1:2 Weights}} & \multicolumn{2}{c|}{\textbf{1:10 Weights}} & \multicolumn{2}{c|}{\textbf{1:100 Weights}} \\  \hline
\multicolumn{1}{|c|}{\textbf{Algorithms}} & \textbf{False Alarm (\%)} & \textbf{Detection (\%)} & \textbf{False Alarm (\%)} & \textbf{Detection (\%)} & \textbf{False Alarm (\%)} & \textbf{Detection (\%)}  \\ \hline
\multicolumn{1}{|c|}{Linear Regression} & 0.xxx & 0.xxx & 0.xxx & 0.xxx & 0.xxx & 0.xxx  \\ \hline
\multicolumn{1}{|c|}{Logistic Regression} & 0.xxx & 0.xxx  & 0.xxx & 0.xxx & 0.xxx & 0.xxx  \\ \hline
\multicolumn{1}{|c|}{Random Forest} & 0.xxx & 0.xxx  & 0.xxx & 0.xxx & 0.xxx & 0.xxx  \\ \hline
\end{tabular}
\caption{False Alarm and Detection Results from Varying Weights and Algorithms}
\end{table}
\subsection{Enterprise-scale deployment}
\section{Conclusion}

\bibliographystyle{acm}
\bibliography{bibliography}

\end{document}
