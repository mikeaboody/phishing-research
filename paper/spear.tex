\documentclass[letterpaper]{article}
\usepackage{url,multirow}
\usepackage{hyperref}
\usepackage{subscript}
\usepackage{cite}
\usepackage{graphicx}
\usepackage{colortbl, xcolor}
\usepackage{todonotes}
\usepackage{caption}
\usepackage{subcaption}
\usepackage{graphicx}
\usepackage{booktabs}
\usepackage{comment}
\usepackage{flushend}
\usepackage{xspace}
\usepackage{booktabs}
\usepackage{mdwlist}
\usepackage{array}
\usepackage[justification=centering]{caption}

\newcolumntype{N}{>{\centering\arraybackslash}m{.7in}}
\newcolumntype{G}{>{\centering\arraybackslash}m{2in}}

\newcommand*{\boldsymbol}[1]{#1}
\newcommand\tab[1][1cm]{\hspace*{#1}}

\begin{document}

\title{title}
\author{authors}
\maketitle

\begin{abstract}
abstract
\end{abstract}

\section{Introduction}
\section{Problem statement}
\section{Our scheme}
\subsection{Architecture/pipeline}
The system we designed to predict incoming emails as either spear phishing or legitimate is divided into 4 parts.

\subsubsection{Logging previously seen emails}
Our system utilizes past emails from the sender to build a template for what we expect a legitimate email from that sender to look like. The first step of the pipeline extracts email headers from packet captures of network traffic. We use the Bro Network Security Montior for this task, extracting the email headers and organizing each email we detect by sender name and email address. For each email, we store the name of the header and the value of the header in the order that we receive them.

\subsubsection{Generating pseudo-spear phishing emails}
We want to provide our classifier with training data that includes both legitimate and spear phishing emails. Because we do not know whether the logged network traffic contains legitimate or spear phishing emails, we generate spoofed spear phishing emails to train our classifier on. The algorithm we use to spoof spear phishing emails is as follows: for every email that we saw from step 1, we choose a second email uniformly at random without replacement from the same set of emails and replace the From header of the second email with the From header of the first email, and store this spoofed email as a spear phishing email.

\subsubsection{Feature extraction on training data}
After step 2, we have a set of legitimate emails and a set of spoofed spear phishing emails. Each email is stored as a set of email headers. We divide these emails into 2 categories: 
\begin{enumerate}
\item Emails used to create a sender profile for the heuristics used to generate our training features
\item Emails used as training data for our classifier.
\end{enumerate} 

Features for the training data are generated from detectors that each focus on a subset of the email headers. These detectors (See Section 3.2), rely on having a sender profile for each sender. The sender profile is a mapping from senders to some characteristic of their past emails. For example, the Timezone detector maps each sender to a list of all time zones that this sender has previously sent email from. We build the sender profile for each detector using the emails in category 1.

Once the sender profile for every detector has been built, we take the emails in category 2 and convert each email into a row of our training data matrix. Each detector takes an email and outputs a score based off of a heuristic function that relies on the sender profile, and each of these scores becomes one feature in our data matrix.

\subsubsection{Train classifier and evaluate on test data}
We use this extracted training data to train a logistic regression classifier. For any new email that we see, we use this classifier to predict the probability that this new email is a spear phishing email.
\subsection{Detectors}
\subsubsection{X-Mailer header}
\tab The X-Mailer header denotes the type of software or mail client that a sender used to send the message. The value of the X-Mailer header often depends on the device used to send the message (and not just the domain part of the email address). For example, an email address with a domain part of	``gmail.com'' might have as its corresponding X-Mailer value ``iPhone Mail (12B466)''. Since the software and device used is a fundamental component of a sender's profile, it has the potential to act as sort of identity marker of a sender. As such, the detector constructed for this header relies on the consistency of the field throughout emails sent by the same sender.\\
\tab The detector's sender profile keeps track of a mapping from senders (email addresses extracted from the From header) to a list of parsed X-Mailer values. However, since email clients update versions frequently, the detector attempts to take out any version numbers. For example, the parsed value of the X-Mailer value ``iPhone Mail (12B466)'' would be ``iPhone Mail''. In addition, if no such X-Mailer header exists in the email, the sender profile will instead append the python value ``None'' to the list of X-Mailer values. With this sender profile, the detector then is able to classify emails by the following process:

\begin{enumerate}
\item Extract the email address from the From header and the parsed value from the X-Mailer value (which is ``None'' if no such X-Mailer header exists in the email's headers).
\item Look up in the sender profile the corresponding sender's list of parsed X-Mailer values.
\item If the parsed X-Mailer value of the email is in the corresponding sender's list of parsed X-Mailer values (ignoring case in comparison), the email is classified as safe.
\item Otherwise, it is classified as phishing.
\end{enumerate}

\tab In the email dataset used in the Evaluation section, it was found that 97.75\% of senders remained consistent with their X-Mailer headers over time (i.e. the list tracked in the sender profile for a given sender had only one unique value of an X-Mailer in it).  Through this classification method and consistency, the detector exploits any lack of consistency of the X-Mailer header for a given sender.  However, the effects of the detector are limiting due to the sparseness of X-Mailer headers in emails (roughly 83.56\% of emails in the dataset used in the Evaluation). Since only 16.43\% of emails have X-Mailer headers, the detector will falsely classify many emails as safe when indeed they are phishing. For this reason, the detection rate (as well as the false alarm rate) is significantly low. Thus, the detector is mostly effective when it is classifying emails from senders that send emails that generally contain X-Mailer values, where the percentage of senders that consistently send the same X-Mailer is 97.05\%.

\subsubsection{Order of headers}

\subsubsection{Received Headers}
The Received header is a trace field prepended to an email message by an SMTP server. An email can contain many Received headers, one for every STMP server it goes through. According to the received header protocol, the Received header typically has the following format:\\

Received: from x.y.test\\
\tab by example.net\\
\tab via TCP\\
\tab with ESMTP\\
\tab id ABC12345\\
\tab for <mary@example.net>;  21 Nov 1997 10:05:43 -0600\\

The Received header detector specifically uses the "from" part of the Received header, which contains a domain name and/or an IP address. With this information, we are able to construct a path of SMTP servers that the email took to get from the sender to the recipient. 

From each Received header, we first want to extract the domain name and IP address from the "from". Since there are many IP addresses that can map to SMTP servers in the same domain, we attempt to find a uniform way of finding out where this particular Received header came from. Therefore, we represent the domain names and/or IP addresses in the "from" as the CIDR blocks they are a part of. If the "from" contains an domain name, we do a WhoIs lookup on the last portion of the domain name. For example, if the domain name is "123.abc.com", we would do a WhoIs lookup on "abc.com". We use the results from the WhoIs lookup to find the CIDR block that this domain is associated with. If the "from" doesn't contain a valid domain, but has an IP address that is not private, we do a reverse DNS lookup to get the domain that this IP address maps to and then perform the same WhoIs lookup process as before. If the "from" does not have a valid domain name or a valid (public) IP address, we assign a value of "Invalid" to this Received header. If this received header does not have a "from" to begin with, we assign a value of "None" to this Received header. Using this processing method of assigning a Received header to a CIDR block, we are able to create a "path", or an ordered list of these values, for an email. 

For the sender profile of this detector, we map each sender to a list of "paths", ordered lists of CIDR blocks for each email sent by the sender. Let's say Alice sends an email to Bob. One such "path" in Bob's sender profile for Alice looks like the following: ["None", 1.2.3.4/16, "Invalid", 1.2.3.4/16, 6.7.8.9/24]. The following is the process we use when classifying an email:\\

\begin{enumerate}
\item We construct the "path" that the Received headers take, based on the process described above.
\item We process the "From" header of the email to find the associated list of paths in the sender profile.
\item We use an edit distance algorithm with a certain thresholds (0, 1, and 2) to measure the similarity of paths. For each threshold, we check to see if the path for this email is similar to a path currently in the sender profile for this sender. If a path is similar to one that is currently in the sender profile for this sender, we classify it as safe, otherwise, we classify it as phishing. This is done for each threshold. 
\end{enumerate}

There are a couple of limitations to this detector. One such limitation is the fact that not all Received headers have a "from". This results in having a "None" in the place of a CIDR block in some paths. Another limitation is that some "from" fields don't have a valid domain name or a public IP address that we could use to find the CIDR block. In this case, we have an "Invalid" in the place of a CIDR block. While our Received header detector mainly focuses on using the "from" field of the Received header, there is a lot more information that can be exploited. For example, we can analyze the "date" field and perform a similar classification as described in the Date header section of this paper.\\

TODO: Results

\subsubsection{Message-ID header}

The Message-ID header is a unique identifier for an email. An example Message-ID header looks like this:\\

Message-ID: $<$123456789.1357911.9876543210987.mail.abc@mail.abc.com$>$\\

This header can be split into two parts using the "@" symbol. The part of the header before the "@" is the ID and the part after the "@" is the domain. We have created two detectors for the Message-ID, one for the ID part of the Message-ID and the other for the domain part of the Message-ID.\\

\textbf{Message-ID: ID}\\

\textbf{Message-ID: Domain}\\

The domain of the Message-ID refers to the domain of the email address that this email has been sent from. Usually, the domain in the Message-ID corresponds to the domain of the email address that an email was sent from. For example, if an email came from "alice@abc.com", a possible Message-ID domain could be "mail.abc.com". The Message-ID detector exploits this relationship to identify suspicious Message-ID domains.

The sender profile for this detector keeps a mapping between the sender and a list of partial Message-ID domains it has seen. Partial, in this case, means that we only add the last part of the Message-ID domain to the sender profile. For example, if a Message-ID domain was "mail.abc.com", we would store "abc.com". This detector also keeps a global mapping of email address domains and Message-ID domains. All partial Message-ID domains that were seen with a particular email address domain are stored in this global mapping. Using this sender profile, we classify emails as follows:\\

\begin{enumerate}
\item Extract the Message-ID domain of an email and only keep the last part (as described above).
\item Check to see if this Message-ID domain is in the sender profile for this sender. If the Message-ID domain is in the sender profile of this sender, we classify the email as safe. If the Message-ID domain is not in the sender profile of this sender, we check to see if this domain is in the global mapping for the email address domain. If it is in this global mapping, then we classify the email as safe.
\item Before an email is classified as phishing, there is one additional check we perform on the Message-ID domain. We do a WhoIs lookup on the last part of the email address domain and we do a WhoIs lookup on the last part of the Message-ID domain. If the CIDR blocks or Company Name obtained from these two WhoIs lookups are the same, we classify the email as safe. The reason we compare this information is because there are certain groups of domains that should not be considered as suspicious. For example, if an email address domain is "gmail.com", and the last part of a Message-ID domain is "google.com", we do not want to classify this email as suspicious because "gmail.com" and "google.com" both belong to the same company.
\end{enumerate}

TODO: Results


\subsubsection{Date header}
\subsubsection{Timezone}
\subsubsection{Content-Type header}
\subsubsection{Content-Transfer-Encoding header}
\section{Evaluation}
\subsection{Personal emails}
[Michael]

\begin{table}[]
\centering
\begin{tabular}{N|N|N|N|N|N|N|}
\cline{2-7}
     & \multicolumn{2}{c|}{\textbf{1:2 Weights}} & \multicolumn{2}{c|}{\textbf{1:10 Weights}} & \multicolumn{2}{c|}{\textbf{1:100 Weights}} \\  \hline
\multicolumn{1}{|c|}{\textbf{Features}} & \textbf{False Alarm (\%)} & \textbf{Detection (\%)} & \textbf{False Alarm (\%)} & \textbf{Detection (\%)} & \textbf{False Alarm (\%)} & \textbf{Detection (\%)}  \\ \hline
\multicolumn{1}{|c|}{X-Mailer} & 0.xxx & 0.xxx & 0.xxx & 0.xxx & 0.xxx & 0.xxx  \\ \hline
\multicolumn{1}{|c|}{Received} & 0.xxx & 0.xxx  & 0.xxx & 0.xxx & 0.xxx & 0.xxx  \\ \hline
\multicolumn{1}{|c|}{Order of Headers} & 0.xxx & 0.xxx  & 0.xxx & 0.xxx & 0.xxx & 0.xxx  \\ \hline
\multicolumn{1}{|c|}{Content Type} & 0.xxx & 0.xxx  & 0.xxx & 0.xxx & 0.xxx & 0.xxx  \\ \hline
\multicolumn{1}{|c|}{Date Format} & 0.xxx & 0.xxx  & 0.xxx & 0.xxx & 0.xxx & 0.xxx  \\ \hline
\multicolumn{1}{|c|}{Message ID Domain} & 0.xxx & 0.xxx  & 0.xxx & 0.xxx & 0.xxx & 0.xxx  \\ \hline
\multicolumn{1}{|c|}{Date Timezone} & 0.xxx & 0.xxx  & 0.xxx & 0.xxx & 0.xxx & 0.xxx  \\ \hline
\end{tabular}
\caption{False Alarm and Detection Results from Varying Weights and Features}
\end{table}


\begin{table}[]
\centering
\begin{tabular}{N|N|N|N|N|N|N|}
\cline{2-7}
     & \multicolumn{2}{c|}{\textbf{1:2 Weights}} & \multicolumn{2}{c|}{\textbf{1:10 Weights}} & \multicolumn{2}{c|}{\textbf{1:100 Weights}} \\  \hline
\multicolumn{1}{|c|}{\textbf{Algorithms}} & \textbf{False Alarm (\%)} & \textbf{Detection (\%)} & \textbf{False Alarm (\%)} & \textbf{Detection (\%)} & \textbf{False Alarm (\%)} & \textbf{Detection (\%)}  \\ \hline
\multicolumn{1}{|c|}{Linear Regression} & 0.xxx & 0.xxx & 0.xxx & 0.xxx & 0.xxx & 0.xxx  \\ \hline
\multicolumn{1}{|c|}{Logistic Regression} & 0.xxx & 0.xxx  & 0.xxx & 0.xxx & 0.xxx & 0.xxx  \\ \hline
\multicolumn{1}{|c|}{Random Forest} & 0.xxx & 0.xxx  & 0.xxx & 0.xxx & 0.xxx & 0.xxx  \\ \hline
\end{tabular}
\caption{False Alarm and Detection Results from Varying Weights and Algorithms}
\end{table}
\subsection{Enterprise-scale deployment}
\section{Conclusion}

\bibliographystyle{acm}
\bibliography{bibliography}

\end{document}
